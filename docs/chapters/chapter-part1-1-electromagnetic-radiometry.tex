\label{chapter1}
\chapter{Richiami sull'elettromagnetismo e Radiometria}
\section{Radiometria}
\label{radiometria}
Branche dell'ottica prese come riferimento dal physically based rendering sono la \textit{Radiometria} e \textit{Radiative Transfer}.
La \textit{Radiometria} \`e il modello matematico per studiare e misurare la propagazione delle radiazioni elettromagnetiche. In tale modello, ci 
disinteressiamo degli aspetti ondulatori delle onde elettromagnetiche e guardiamo la radiazione come energia fluente nello spazio. Ci\`o suggerisce
che la Radiometria opera a livello dell'ottica geometrica, assumendo dunque propagazione rettilinea della luce e trascurando fenomeni dovuti alla 
natura ondulatoria della radiazione. Si considerado solamente gli effetti della riflessione, trasmissione, scattering ed assorbimento.
In particolare, le assunzioni fatte sono le seguenti[\cite{pharr}]:
\begin{altDescription}{chapter1:radiometria-assunzioni}
	\item[Linearit\`a] La combinazione lineare di pi\`u input ad un sistema ottico \`e pari alla combinazione lineare degli effetti dei singoli input
	\item[Conservazione dell'energia] Quando una fonte di radiazioni scatters da una superficie o mezzo trasmissivo, gli \textit{scattering events}
		[\cite{pegoraro}], non possono produrre pi\`u energia di quella iniziale
	\item[No polarizzazione] Ignoriamo la polarizzazione del campo elettromagnetico di una radiazione, rendendo l'unica propriet\`a rilevante di
		una radiazione la sua \textit{Distribuzione Spettrale} [\nameref{chapter2}]
	\item[No fluorescenza o fosforescenza] Il comportamento della radiazione ad una lunghezza d'onda \`e indipendente dal comportamento della 
		radiazione ad altre lunghezze d'onda
	\item[A regime] Il \textit{Light Field} (\nameref{chapter1:radianza}) nell'ambiente \`e assunto a regime, indipendente dal tempo. Nota che la 
		fosforescenza viola anche questo presupposto
\end{altDescription}
Il \textit{Radiative Transfer}, qui solo accennato, \`e lo studio dei fenomeni legati al trasferimento di energia radiante, basato sulla radiometria,
ed utilizzato nella computer grafica per introdurre fenomeni legati all'ottica fisica e modellare la propagazione in un mezzo diverso dal vuoto.\par
Le unit\`a fondamentali della radiometria sono riassunte nella tabella \ref{chapter1:radiometric-quantities}
\begin{table}[tb]
	\begin{tabularx}{\linewidth}{cccY}
		\toprule
		Quantit\`a & Simbolo & unit\`a S.I. & Note \\
		\midrule
		\Gls{Energia Radiante} & $Q_e$ & \si{J} & Energia aggregata di tutti i fotoni in una radiazione elettromagnetica\\
		\Gls{Flusso Radiante}  & $\Phi_e$ & \si{W} & Energia Radiante \textit{emessa, riflessa, trasmessa, incidente}, per unit\`a di tempo\\
		\Gls{Intensita Radiante} & $I_{e,\Omega}$ & \si{W/sr} & Flusso Radiante emesso, riflesso, trasmesso, per uni\`a di angolo solido\\
		\Gls{Irradianza} & $E_e$ & \si{W/m^2} & Flusso Radiante incidente per unit\`a di area\\
		\Gls{Emittanza Radiante} & $M_e$ & \si{W/m^2} & Flusso Radiante emesso per unit\`a di area\\
		\Gls{Radiosita} & $J_e$ & \si{W/m^2} & Flusso Radiante \textit{uscente} (emesso, riflesso e trasmesso) per unita di area\\
		\Gls{Radianza} & $L_{e,\Omega}$ & \si{W/(sr.m^2)} & Flusso Radiante emesso, riflesso, trasmesso, incidente da/su una superficie, per unit\`a di 
			angolo solido per unita di \textit{area proiettata}\\
		\bottomrule
	\end{tabularx}
	\caption{nomenclatura e misure delle quantit\`a radiometriche per noi rilevanti}
	\label{chapter1:radiometric-quantities}
\end{table}
\begin{definitionS}
	L'\textit{Energia Radiante} $Q_e$ \`e l'energia aggregata di tutti i fotoni, ognuno ad una particolare lunghezza d'onda con contributo energetico
	diverso, \textit{verso/attraverso/da} una superficie in un dato intervallo di tempo\par
	Ciascuno di questi fotoni trasporta energia pari all'Equazione di plank-Einstein $Q_\lambda=\frac{hc}{\lambda}$
\end{definitionS}
\begin{definitionS}
	L'\textit{Energia Radiante Spettrale} $Q_{e,\lambda}$\footnotemark{} \`e il contributo di una singola lunghezza d'onda all'energia radiante 
	complessiva
	\[ Q_{e,\lambda}= \frac{\partial Q_e}{\partial\lambda} \]
\end{definitionS}
\footnotetext{Le grandezze spettrali possono essere espresse in funzione della lunghezza d'onda $\lambda$ ( e.g. $Q_{e,\lambda}$ ) o del suo inverso, la
	frequenza ( e.g. $Q_{e,\nu}$ )}
Data l'assunzione di sistema a regime, siamo interessati a misurare l'energia radiante in un istante piuttosto che per un periodo definito di tempo.
Dunque definiamo
\begin{definitionS}
	Il \textit{Flusso Radiante} $\Phi_e$ \`e la quantit\`a di energia radiante passante attraverso una superficie per unit\`a di tempo
	\[ \Phi_e = \frac{\partial Q_e}{\partial t} \]
	La cui derivata rispetto alle frequenze fornisce il \textit{Flusso Radiante Spettrale}
	\[ \Phi_{e,\lambda} = \frac{\partial\Phi_e}{\partial\lambda} \]
\end{definitionS}
Ogni misurazione di un flusso richiede un area ben definita per attribuirle senso. Nei sistemi di rendering, non \`e insolito invece compiere delle 
misure "puntuali" di un flusso radiante, cio\`e \textit{flusso radiante per unit\`a di area}.
\begin{definitionS}
	La densit\`a superficiale del flusso radiante (energia per unit\`a di tempo, per unit\`a di superficie) \`e detta \textit{Irradianza} $E_e$ se tale 
	flusso \`e entrante la superficie, \textit{Emittanza Radiante} $M_e$ se emesso dalla superficie, \textit{Radiosit\`a} $J_e$ se uscente (emesso, 
	riflesso o trasmesso) dalla superficie
	\[ E_e | J_e | M_e = \frac{\partial\Phi_e}{\partial A} \]
	La cui derivata rispetto alle frequenze fornisce la rispettiva grandezza spettrale
	\[ E_{e,\lambda} | J_{e,\lambda} | M_{e,\lambda} = \frac{\partial E_e | J_e | M_e}{\partial\lambda} \]
	Si noti infine che l'Irradianza, per definizione, \`e pari alla media temporale in un periodo del vettore di Pointing perpendicolare alla 
	superficie \[ E_e = \langle |\vec{S}| \rangle \]
\end{definitionS}
Piuttosto che concentrarci su un unico punto della superficie, possiamo concentrarci su una singola direzione, considerando il flusso radiante per 
unit\'a di angolo solido.
\begin{definitionS}
	La \textit{Intensit\`a Radiante} $I_{e,\Omega}$ \`e la densit\`a angolare del flusso radiante emesso, riflesso, trasmesso o incidente da/verso un 
	punto dello spazio e propagante lungo una direzione specificati
	\[ I_{e,\Omega} = \frac{\partial\Phi_e}{\partial\Omega} \]
	L'\textit{Intensit\`a (Radiante) Spettrale} \`e l'intensit\`a radiante per unit\`a di lunghezza d'onda
	\[ I_{e,\Omega,\lambda} = \frac{\partial I_{e,\Omega}}{\partial\lambda} \]
\end{definitionS}
Tale grandezza \`e utile per la modellazione, ad esempio, delle sorgenti luminose puntuali omnidirezionali, le quali distribuiscono il loro 
flusso radiante emesso, inversamente proporzionale al quadrato della distanza dalla sorgente, in modo uniforme il ogni direzione 
$I_{e,\Omega} = \frac{\Phi_e}{4\pi}$.\par
La descrizione pi\`u granulare dell'energia emessa da un'onda elettromagnetica \`e fornita dalla \textit{Radianza}, definita come
\label{chapter1:radianza}
\begin{definitionS}
	La \textit{Radianza} $L_{e,\Omega}$ \`e definita come flusso radiante emesso, riflesso, trasmesso o incidente da/su una dato punto della 
	superficie, da/verso una data direzione, dunque per unit\`a di angolo solido per unit\`a di area \textit{proiettata}
	\[ L_{e,\Omega} = \frac{\partial^2\Phi_e}{\partial\Omega\partial(A\cos\theta)} \]
	dove $\theta$ angolo tra la direzione perpendicolare alla superficie sul punto considerato e la direzione del flusso
\end{definitionS}
Significativa ai nostri fini \`e la relazione inversa che sussiste tra Irradianza e Radianza, che sfrutteremo per definire l'equazione del rendering
\begin{align}
	E_e(\vec{p}) &= \int_\Omega L_{e,\Omega}(\vec{p},\hat{\omega})\cos\theta\mathrm{d}\hat{\omega}\footnotemark 
				 &= \int_{\theta_1}^{\theta_2} \int_{\varphi_1}^{\varphi_2} L_{e,\Omega}(\vec{p}, \theta, \varphi)
						\cos\theta\sin\theta\mathrm{d}\theta\mathrm{d}\varphi
\end{align}
\footnotetext{Si noti la distinzione tra $\hat\omega$ e $\mathrm{d}\hat\omega$. Il primo \`e un vettore unitario dal punto $\vec{p}$ ad un 
	punto della porzione di sfera unitaria costituente il dominio di integrazione, il secondo \`e un elemento di angolo angolo solido. 
	Un angolo solido \`e  un  numero adimensionale rappresentante la porzione di area di sfera coperta, nell'intervallo $[0,4\pi]$.
	Il suo differenziale rappresenta un piccolo quadratino della superficie della sfera, spesso approssimato ad un quadrato ai fini 
	di dimostrazioni geometriche}
\subsection{Radianza incidente ed uscente}
La radianza $L$ \`e generalmente una funzione discontinua sulle superfici di separazione tra due mezzi, dette interfacce.
\begin{align}
	L^+ &= \lim_{t\rightarrow 0^+} L(\vec{p}+t\hat{n},\hat{\omega})\\
	L^- &= \lim_{t\rightarrow 0^-} L(\vec{p}+t\hat{n},\hat{\omega})
\end{align}
dunque, piuttosto che lavorare con un unica funzione discontinua, si preferisce distinguere tra \textit{radianza incidente} $L_i$ e 
\textit{radianza uscente}\footnote{emessa, riflessa, trasmessa}.
\begin{align}\label{chapter1:radiometry:incidentOutgoingRadiances}
	L_i(\vec{p},\hat{\omega})&=\left\{
	\begin{aligned}
		L^+(\vec{p},-\hat{w}),\text{ }\langle\omega,\hat{n}\rangle>0\\
		L^-(\vec{p},-\hat{w}),\text{ }\langle\omega,\hat{n}\rangle<0
	\end{aligned}\right.\footnotemark{}\\
	L_o(\vec{p},\hat{\omega})&=\left\{
	\begin{aligned}
		L^+(\vec{p},\hat{w}),\text{ }\langle\omega,\hat{n}\rangle>0\\
		L^-(\vec{p},\hat{w}),\text{ }\langle\omega,\hat{n}\rangle<0
	\end{aligned}\right.
\end{align}
\footnotetext{Si noti come per convenzione, la radianza incidente la direzione $\hat{\omega}$ sia uscente dal punto $vec{p}$, nonostante 
	il flusso radiante abbia direzione opposta. In altre parole, la radianza incidente ha come argomento la direzione di provenienza del flusso}
Si noti che in punti dello spazio non giacenti su una superficie, la radianza \`e invece continua, $L^+=L^-$, da cui
\begin{equation}
	L_o(\vec{p},\hat{\omega}) = L_i(\vec{p},-\hat{\omega})
\end{equation}
Si pu\`o riscrivere la conservazione della radiaza di base in termini della radianza incidente e radianza uscente
\begin{equation}
	\frac{L_o(\vec{p}, \hat{\omega})}{\eta^2_1} = \frac{L_i(\vec{q}, \hat{\omega})}{\eta^2_2}
\end{equation}
\section{Radiazione di Corpo Nero}
Tutti i corpi con temperatura superiore allo zero assoluto emettono radiazione, in quantit\`a proporzionale alla temperatura assoulta in quanto essa 
corrisponde allo stato di agitazione delle molecole, causando accelerazione di elettroni e protoni degli atomi costituenti, portando all'emissione 
di campo elettromagnetico secondo le equazioni di Maxwell. Tale flusso radiante spettrale emesso costituisce una \textit{Spectral Power Distribution}
(SPD). \par
Un \textit{Corpo Nero} corpo ideale che assorbe ogni tipo di radiazione incidente, indipendentemente da frequenza o angolo di incidenza. In
equilibrio termico, esso emette radiazioni che seguono la \textit{Legge di Planck}\ref{chapter1:planckLaw}. Esso \`e un 
\textit{Radiatore Ideale e diffusore}, cio\`e emette radiazioni uniformemente in tutte le direzioni (superficie lambertiana).\par
per tali corpi neri, l'Emittanza radiante $M_{bb}$ dalla superficie obbedisce alla \textit{Legge di Stefan-Boltzmann}
\begin{equation}
	M_{bb} = \sigma T^4
\end{equation}
dove $\sigma = 5.67032\cdot 10^{-8} \si{W/(m^2 K^4)}$ costante di Stefan-Boltzmann\\
Dunque tale corpo nero \`e una superficie opaca (trasmittanza nulla), perfettamente assorbente (absorptance unitaria) che non riflette alcuna radiazione
incidente (riflettanza nulla). Ci\`o vuol dire che emette tutta la radiazione che assorbe ($A+R+T=1$).\par
Essendo una superficie lambertiana, anche la radianza emessa deve essere constante. Sapendo che un emisfera unitaria ha integrale pari a $\pi$,
\begin{equation}
	M_{bb\lambda} = \pi L_{bb\lambda}
\end{equation}
In equilibrio termico, tale radianza ha espressione detta \textit{Legge di Planck}
\begin{equation}\label{chapter1:planckLaw}
	L_{bb\lambda}(T,\lambda) = \frac{2hc^2}{\lambda^5\left(e^{\frac{hc}{\lambda kT}}-1\right)}
\end{equation}
dove
\begin{align}
	h &= 6.626176\cdot 10^{-34} \si{J/s}\text{ costante di Planck}\\
	k &= 1.380662\cdot 10^{-23} \si{J/K}\text{ costante di Boltzmann}
\end{align}
\`E facile dimostrare come, integrando nell'emisfera unitaria, si ottiene un emittanza radiante corrispondente alla legge di Stefan-Boltzmann\par
Tali corpi neri si dimostrano utili non solo come modelli per parametrizzare, tramite temperatura, una sorgente luminosa, ma risultano rilevanti 
anche per la colorimetria, in quanto costituenti il luogo plankiano.\par

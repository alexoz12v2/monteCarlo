\documentclass[12pt, openany]{book}

\usepackage[T1]{fontenc}
\usepackage{savetrees}
\usepackage{tgbonum}
\usepackage[inline]{enumitem}
\usepackage[dvipsnames]{xcolor}
\usepackage{fancyhdr}
\usepackage{hyperref}
\usepackage{graphicx}
\usepackage{lmodern} % more font sizes

% package to format advanced mathematics
\usepackage{amsmath, amssymb, amsthm}
\theoremstyle{definition}
\newtheorem{definition}{Def.}
\newtheorem{theorem}{Th. }
% packaeg needed to format lists in a custom defined format called description

\usepackage{calc}
\newenvironment{altDescription}[1]
	{\begin{list}{}%
		{\renewcommand\makelabel[1]{\textsf{##1:}\hfil}%
			\settowidth\labelwidth{\makelabel{#1}}%
			\setlength\leftmargin{\labelwidth+\labelsep}}}
	{\end{list}}

% text over math symbols
\usepackage{mathtools}
\usepackage{stackrel}

% package which can create text boxes which span over multiple pages
\usepackage{mdframed}

% package which can combine columns of a table
\usepackage{multirow}

% package to extend tables
\usepackage{array}

%graphics package
\usepackage{tikz}
\usetikzlibrary{mindmap}

%slanted roman differentials, following ISO-31
\newcommand*\dif{\mathop{}\!\textnormal{\slshape d}}

% default font times new roman
\renewcommand{\familydefault}{\rmdefault}

% shortcut to typeset code
\newcommand{\code}[1]{{\fontsize{12}{12}\fontfamily{qcr}\selectfont{#1}}}
\newcommand\tab[1][1cm]{\hspace*{#1}}

% sizes to better display footnotes
\setlength{\headheight}{15pt}
\setlength{\footskip}{15pt}
\setlength{\headheight}{15pt}
\setlength{\footskip}{15pt}

% footer setup using fancyhdr package
\pagestyle{fancy}
\fancyfoot{} % to clear previous definitions of footer format
\fancyfoot[LE,RO]{\thepage}

\begin{document}
I metodi deterministici per la determinazione degli integrali multidimensionali, come i metodi di Newton-Cotes, tra cui la regola del rettangolo, 
regola del trapezio, formula $\frac{2}{3}$ di Simpson e formula $\frac{3}{8}$ di Simpson, quadratura gaussiana, subiscono la cosiddetta 
\textit{curse of dimensionality}. Ci\`o significa che il loro tasso di convergenza diminuisce in modo notevole e complessit\`a computazionale aumenta
in modo esponenziale con la dimensionalit\`a del problema.\par
Gli algoritmi randomizzati, in particolare i \textit{Metodi di Monte Carlo} per l'integrazione numerica offrono una soluzione a tale problema.
Prima di tutto, definiamo con rigore tali classi di algoritmi
\begin{definition}
    Un \textsl{\textbf{Algoritmo Randomizzato}} \`e un algoritmo che implementa un certo grado di casualit\`a come parte della sua logica.
    Tali algoritmi si dividono in due classi: \textsl{\textbf{Las Vegas Algorithms}} e \textsl{\textbf{Monte Carlo Algorithms}}
\end{definition}
La differenza tra queste due categorie sta nel fatto che nella prima, \`e garantita la correttezza dell'output della procedura, mentre un algoritmo 
appartenente alla seconda categoria pu\`o sia restituire risultati incorretti oppure non terminare. Ci\`o nonostante, in alcuni casi questi ultimi
rappresentano l'unica soluzione fattibile in termini di tempo e risorse.\par
La propriet\`a dell'integrazione di Monte Carlo \`e la sua capacit\`a di stimare l'integrale $\int f(x)\dif{x}$ valutando la funzione
$f(x)$ in alcuni punti del suo dominio. Tale propriet\`a risulta determinante nelle situazioni in cui non si conosce o \`e impossibile conoscere la
forma chiusa della soluzione analitica del problema, il che capita nella maggior parte degli integrali che emergono nel rendering, come l'integrale
per la computazione della radianza emessa da un singolo punto di una superficie:
\begin{displaymath}
    L_o\left(\mathbf{x}, \omega_o, \lambda, t\right) = L_e\left(\mathbf{x}, \omega_o, \lambda, t\right) + %
        \int_\Omega f_r\left(\mathbf(x), \omega_i, \omega_o, \lambda, t\right)L_i\left(\mathbf{x}, \omega_i, \lambda, t\right) %
        \left(\omega_i \cdot \mathbf{n}\right)\dif{\omega_i}
\end{displaymath}
per la quale non conosciamo nemmeno l'espressione analitica della funzione integranda della radianza incidente $L_i$, ed anche se fosse nota, 
l'integrale non sarebbe facilmente risolvibile.\par
I maggiori svantaggi dell impiego dell'Integrazione di Monte Carlo sono
\begin{altDescription}{svantaggi}
    \item[\textit{Tasso di Convergenza}] se n campioni sono utilizzati per la valutazione dell'integrale, l'algoritmo converge come 
        $\mathit{O}(n^{-1/2})$, il che significa che per dimezzare l'errore bisogna quadruplicare il numero di samples.
    \item[\textit{Rumore}] Tipico dell'applicazione dell'algoritmo \`e la presenza del rumore nelle immagini prodotte, che si manifesta come
        pixels troppo luminosi o troppo scuri.
\end{altDescription}
\end{document}

\documentclass[14pt, aspectratio=169]{beamer}

% ------------------ uncomment to produce handout
%\documentclass[handout,14pt, aspectratio=169]{beamer}
%\usepackage{pgfpages}
%%\pgfpagesuselayout{4 on 1}[a4paper,border shrink=5mm]
%\mode<handout>{
%  \usepackage{pgfpages}
%  \pgfpagesuselayout{4 on 1}[a4paper,landscape,border shrink=5mm]
%  \usepackage{tikz}
%  \setbeamertemplate{background canvas}{
%    \begin{tikzpicture}[remember picture]
%      \draw (current page.south west) rectangle (current page.north east);
%    \end{tikzpicture}
%  }
%}

% pseudocode first slide
\usepackage{algorithm}
\usepackage{algpseudocode}

\algnewcommand\algorithmicinput{\textbf{Input:}}
\algnewcommand\Inputt{\item[\algorithmicinput]}

\algnewcommand\algorithmicoutput{\textbf{Output:}}
\algnewcommand\Outputt{\item[\algorithmicoutput]}

% -------------------- beamer configuration
% theme
\usetheme{Copenhagen}
\usecolortheme{beaver}

% hide navigation symbols
\setbeamertemplate{navigation symbols}{}

% hide section summary header
\setbeamertemplate{headline}{}

\usepackage{xcolor}

% --------- math packages
% theorems
\usepackage{amsthm}
\usepackage{amsmath}
\usepackage{amssymb}
\usepackage{amsfonts}
\usepackage{upgreek} % uptau
\usepackage{siunitx}
\usepackage{caption}
\usepackage{subcaption}
\usepackage{stackrel}
\usepackage{mathtools} % norm
\usepackage[inline]{enumitem}
\usepackage{fancyhdr}
\usepackage{hyperref}
\usepackage{nameref}
\usepackage{cleveref}
\usepackage{graphicx}
\usepackage{svg}
\usepackage{calc}
\usepackage{mdframed}
\usepackage{environ} % to declare scaletikzpicturetowidth with NewEnviron

% ------------ font related packages
% stuff for utf16
\usepackage{unicode-math-luatex}
\usepackage{unicode-math}
\usepackage{fontspec}
\usepackage{lmodern} % more font sizes

\setmathfont[mathrm=sym]{Latin Modern Math}

% ------------------ page layout
%\usepackage{geometry}
%\geometry{
%	inner=37.125mm,
%	outer=33.4125mm,
%	top=37.125mm,
%	bottom=37.125mm,
%	heightrounded,
%	marginparwidth=51pt,
%	marginparsep=17pt,
%	headsep=24pt,
%}

% ------------- beautiful tables
\usepackage{booktabs}

% -------------- the bibliography
\usepackage[
	backend=biber,
	style=alphabetic,
	sorting=ynt
]{biblatex}

% ------------------- graphics
\usepackage{tikz}
\usepackage{pgfplots}
\usepackage{physics}
\usepackage{ifthen}
\usepackage{cutwin} % wrapfigures with beamer https://tex.stackexchange.com/questions/56205/wrapfigure-beamer-style

\usetikzlibrary{shadings}
\usetikzlibrary{calc}
\usetikzlibrary{arrows,arrows.meta} % Stealth arrows
\usetikzlibrary{angles,quotes} % for \draw pic (angle labels)
\usetikzlibrary{graphs,graphdrawing} % tree drawing for Owen Scrambling
\usegdlibrary{trees}
\tikzset{>=latex} % for LaTex arrow head

% stuff for explained equation
\tikzset{mathterm/.style={draw=black,fill=white,rectangle,anchor=base}}
\tikzstyle{every picture}+=[remember picture]

\newcommand{\professor}{Prof.ssa Marina Popolizio}

\makeatletter
\newsavebox{\measure@tikzpicture}
\NewEnviron{scaletikzpicturetowidth}[1]{%
	\def\tikz@width{#1}%
	\def\tikzscale{1}\begin{lrbox}{\measure@tikzpicture}%
	\BODY
	\end{lrbox}%
	\pgfmathparse{#1/\wd\measure@tikzpicture}%
	\edef\tikzscale{\pgfmathresult}%
	\BODY
}

\newsavebox\myboxA
\newsavebox\myboxB
\newlength\mylenA

\newcommand*\xoverline[2][0.75]{%
	\sbox{\myboxA}{$\m@th#2$}%
	\setbox\myboxB\null% Phantom box
	\ht\myboxB=\ht\myboxA%
	\dp\myboxB=\dp\myboxA%
	\wd\myboxB=#1\wd\myboxA% Scale phantom
	\sbox\myboxB{$\m@th\overline{\copy\myboxB}$}%  Overlined phantom
	\setlength\mylenA{\the\wd\myboxA}%   calc width diff
	\addtolength\mylenA{-\the\wd\myboxB}%
	\ifdim\wd\myboxB<\wd\myboxA%
	\rlap{\hskip 0.5\mylenA\usebox\myboxB}{\usebox\myboxA}%
	\else
	\hskip -0.5\mylenA\rlap{\usebox\myboxA}{\hskip 0.5\mylenA\usebox\myboxB}%
	\fi}

\setbeamertemplate{title page}{
  \vbox{}
  \vfill
  \begingroup
    \centering
    \begin{beamercolorbox}[sep=8pt,center]{title}
      \usebeamerfont{title}\inserttitle\par%
      \ifx\insertsubtitle\@empty%
      \else%
        \vskip0.25em%
        {\usebeamerfont{subtitle}\usebeamercolor[fg]{subtitle}\insertsubtitle\par}%
      \fi%     
    \end{beamercolorbox}%
    \vskip1em\par
    \begin{beamercolorbox}[sep=0cm,wd=\the\@tempdima]{frametitle}
       \begin{tikzpicture}
        \draw[line width = 1.5] (-2,0) -- (12,0);
       \end{tikzpicture}
    \end{beamercolorbox}
    \begin{beamercolorbox}[sep=8pt,right]{author}
      \usebeamerfont{author}{Presentata da: \insertauthor}
    \end{beamercolorbox}
    \vskip1em
    \begin{beamercolorbox}[sep=8pt,left]{author}
      \usebeamerfont{author}{Relatore: \professor}
    \end{beamercolorbox}
    {\usebeamercolor[fg]{titlegraphic}\inserttitlegraphic\par}
    \begin{beamercolorbox}[sep=8pt,left]{institute}
      \usebeamerfont{institute}{\insertinstitute},
      \usebeamerfont{date}\insertdate
    \end{beamercolorbox}
  \endgroup
  \vfill
}

\newcommand*{\currentsectionname}{\@currentlabelname}

\newcommand{\betweensmallandscript}{\@setfontsize\betweensmallandscript{9pt}{10.8pt}}

% Designate a term in a math environment to point to
% Syntax: \mathterm[node label]{some math}
\newcommand\mathterm[2][]{%
   \tikz [baseline] { \node [mathterm] (#1) {$#2$}; }}

% A command to draw an arrow from the current position to a labelled math term
% Default color=black, default arrow head=stealth
% Syntax: \indicate[color]{term to point to}[path options]
\newcommand\indicate[2][black]{%
   \tikz [baseline] \node [inner sep=0pt,anchor=base] (i#2) {\vphantom|};
   \@ifnextchar[{\@indicateopts{#1}{#2}}{\@indicatenoopts{#1}{#2}}}
\def\@indicatenoopts#1#2{%
   {\color{#1} \tikz[overlay] \path[line width=1pt,draw=#1,-stealth] (i#2) edge (#2);}}
\def\@indicateopts#1#2[#3]{%
   {\color{#1} \tikz[overlay] \path[line width=1pt,draw=#1,-stealth] (i#2) [#3] edge (#2);}}

\makeatother

% -------------------------- code highlighting 
\usepackage[outputdir=intermediate]{minted}

% --------------------- empty table captions
\usepackage{caption}
\captionsetup{labelformat=empty}

% ------------------------- colors and other stuff for figures
% EM wavw
\colorlet{myblue}{black!40!blue}
\colorlet{myred}{black!40!red}
\colorlet{vcol}{green!50!black}
\colorlet{Ecol}{orange!90!black}
\colorlet{EVcol}{orange!80!black!60}
\colorlet{Bcol}{violet!90}

% EM spectrum
\colorlet{wavecol}{orange!35!black}
\colorlet{freqcol}{green!25!black}
\colorlet{enercol}{blue!35!black}
\pgfdeclareverticalshading{rainbow}{100bp}{
	color(0bp)=(red); color(25bp)=(red); color(35bp)=(yellow);
	color(45bp)=(green); color(55bp)=(cyan); color(65bp)=(blue);
	color(75bp)=(violet); color(100bp)=(violet)
}

% ------------------- units of measurements consistent style
\sisetup{
	detect-all,
	inter-unit-product=\ensuremath{{}\cdot{}},
  	separate-uncertainty=true,
  	uncertainty-separator={\,},
  	range-units=single,
  	per-mode=symbol,
  	exponent-product=\cdot,
  	output-decimal-marker={,},
}

% ----------------------------------- symbol definitions
% imaginary unit
\newcommand{\iu}{{i\mkern1mu}}

% ---------------- glossaries
\usepackage[toc]{glossaries}

% -------------------------- better management of larger projects
\usepackage{import}

% ----------------- stuff for customized title page
% \usepackage{titlesec}

% -------------------------- tables
\usepackage{tabularx} % equal spacing
\usepackage{tabulary} % spacing proportional to content
\usepackage{xltabular} % multipage tabularx
\usepackage{multirow}

% left-justified text in tables, and allow newlines
\newcolumntype{Y}{>{\raggedright\arraybackslash}X}
\newcolumntype{U}{>{\raggedright\arraybackslash}c}

% bibliography declaration
\addbibresource{references.bib}

% default font times new roman
\renewcommand{\familydefault}{\rmdefault}

% definition environment
\newtheoremstyle{theoremdd}% name of the style to be used
	{\topsep}% measure of space to leave above the theorem. E.g.: 3pt
  	{\topsep}% measure of space to leave below the theorem. E.g.: 3pt
  	{}% name of font to use in the body of the theorem
  	{0pt}% measure of space to indent
  	{\bfseries}% name of head font
  	{. ---}% punctuation between head and body
  	{ }% space after theorem head; " " = normal interword space
  	{\thmname{#1}\thmnumber{ #2}\textnormal{\thmnote{ (#3)}}}
\theoremstyle{theoremdd}
\newtheorem{definitionS}{Def.}[section] % numbered by section
\newtheorem{theoremS}{Th.}[section] % numbered by section

% shortcut to typeset code
\newcommand{\code}[1]{{\fontsize{10pt}{12pt}\fontfamily{qcr}\selectfont{#1}}}
\newfontfamily\codefont{DejaVu Sans}
\newcommand\tab[1][1cm]{\hspace*{#1}}
\newcommand{\littext}[1]{{\color{Rhodamine}{#1}}}
\newcommand{\hemi}[1]{\ensuremath{\mathfrak{H}^2(#1)}}
%\newcommand{\vector}[1]{\ensuremath{\mathbf{\overrightarrow{#1}}}}
\newcommand{\sha}{\ensuremath{\text{ш}}}

\newenvironment{altDescription}[1]
	{\begin{list}{}%
		{\renewcommand\makelabel[1]{\textsf{##1:}\hfil}%
			\settowidth\labelwidth{\makelabel{#1}}%
			\setlength\leftmargin{\labelwidth+\labelsep}}}
	{\end{list}}

\setmainfont{CharisSIL}
[
	Extension=.ttf,
	UprightFont=*-Regular,
	ItalicFont=*-Italic,
	BoldFont=*-Bold,
	BoldItalicFont=*-BoldItalic,
	Path=/usr/share/fonts/TTF/
]
\setmathfont{Latin Modern Math}
\setmathfont{XITSMath-Regular}
[
	Extension = .otf,
	BoldFont = XITSMath-Bold,
	Path=/usr/share/fonts/OTF/,
	range={"03F6,"0410-"044F}
]


\colorlet{myblue}{blue!80!black!70!red!70}
\colorlet{glasscol}{blue!10}
\tikzstyle{glass}=[top color=glasscol!88!black,bottom color=glasscol,middle color=glasscol!98!black,shading angle=0]
\tikzset{
	light beam/.style={
		line width=0.5, 
		arrows={-Stealth[length=2pt, inset=0.5pt, angle=70:4pt]},
		draw=#1,
	},
	light beam/.default=myblue}
\newcommand\rightAngle[4]{
	\pgfmathanglebetweenpoints{\pgfpointanchor{#2}{center}}{\pgfpointanchor{#3}{center}}
	\coordinate (tmpRA) at ($(#2)+(\pgfmathresult+45:#4)$);
	\draw[white,line width=0.6] ($(#2)!(tmpRA)!(#1)$) -- (tmpRA) -- ($(#2)!(tmpRA)!(#3)$);
	\draw[blue!40!black] ($(#2)!(tmpRA)!(#1)$) -- (tmpRA) -- ($(#2)!(tmpRA)!(#3)$);
}

% higher page numbers
\defbeamertemplate{footline}{higher page number}
{%
	\hskip10pt%\vspace{-10pt}
	\includegraphics[width=0.07\linewidth]{./assets/Poliba_stemma.png}
	\hfill%
	\usebeamercolor[fg]{page number in head/foot}%
	\usebeamerfont{page number in head/foot}%
	\insertframenumber\,/\,\inserttotalframenumber\kern1em\vskip10pt%
}
\setbeamertemplate{footline}[higher page number]

\title{Metodi di Monte Carlo Applicati Alla Computer Grafica}
\author{Tanzi Alessio}
\date{Anno Accademico 2022/2023}
\institute[]{Politecnico di Bari}

%\setbeamercovered{transparent} % permette di vedere il contenuto delle prossime slides del frame

\pgfkeys{/tikz/.cd,
	at/.initial={(0,0)},
	at/.get=\coordpos,
	at/.store in=\coordpos,   
	hemi/.code={
		% frame coordinates
		\coordinate (O) at (0,0); % origin
		\coordinate (R) at (2,0); % point to the right
		\coordinate (N) at (0,2); % top of the sphere
		\coordinate (U) at (R);
		\coordinate (V) at (canvas polar cs:angle=240,radius=0.67cm);

		% light rays coordinate
		\def\ang{30}
		\def\projang{15}
		\def\projdist{0.2cm}
		\def\ray{1.7cm}
		\coordinate (I) at (canvas polar cs:angle=90+\ang+\projang, radius=\ray); % view direction
		\coordinate (OR) at (canvas polar cs:angle=90-\ang, radius=\ray-\projdist); % reflected direction

		% angles 
		\draw pic["",draw=black,angle radius=20,angle eccentricity=1.3] {angle = N--O--I} node [anchor=north, above=23, left=0.05] {$\theta_i$};

		% frame arrows
		\draw[light beam=black] (O) -- (V) node [anchor=north east, above left=0.2] {$\hat{v}$};
		\draw[light beam=black] (O) -- (N) node [anchor=east, below left=0.2] {$\hat{n}$};
		\draw[light beam=black] (O) -- (R) node [anchor=north, above left=0.3] {$\hat{u}$};

		% light arrows
		\draw[light beam] (O) -- (I) node [anchor=north west, below=0.1] {$\hat{\omega}_i$};

		% point p
		\draw[black,fill=white] (O) circle [radius=0.05] node [below right=0.07] {$\vec{p}$};

		% the ball
		\shade[ball color=blue!70!gray!20, opacity=.15] 
			(R) arc (0:180:2cm) arc (180:360:2cm and 0.6cm);
			
		% outline of the hemisphere in black
		\draw (R) arc (0:180:2cm) arc (180:360:2cm and 0.6cm); 
		\draw [dashed] (R) arc (0:180:2cm and 0.6cm);
	}
}

\newcommand{\showslide}[3]{
\only<#1>{
	\begin{figure}[tb]
		\begin{subfigure}[c]{0.45\textwidth}
		\centering
		\includegraphics[width=\textwidth]{../assets/appendixD_result_#2.png}
		\caption{#2 campioni per stratum}
		\end{subfigure}
		\hfill
		\begin{subfigure}[c]{0.45\textwidth}
		\centering
		\includegraphics[width=\textwidth]{../assets/appendixD_result_#3.png}
		\caption{#3 campioni per stratum}
		\end{subfigure}
	\end{figure}
}}
\begin{document}
	\pgfplotsset{compat=newest}
	\begin{frame}
		\maketitle
	\end{frame}

	\begin{frame}{Introduzione}
		\begin{columns}[T]
			\begin{column}{0.35\textwidth}
				\includegraphics[width=\linewidth]{../assets/intro_path_tracing_online_tech_tips.png}
				\small Stima della funzione immagine $r_f(x,y)$ attraverso Path Tracing.
			\end{column}
			\begin{column}{0.65\textwidth}
				\only<1-3>{
					\begin{itemize}[topsep=0pt,noitemsep]
						\item[\bullet]<1-> Calcolo della Funzione Immagine
						\item[\bullet]<2-> Attraverso il suo campionamento
						\item[\bullet]<3-> Approssimazione con MC
					\end{itemize}
					\hbox{\hfill}
					\only<1>{\begin{equation*}
						r_f(x,y)=\int_{A_{px}}r(x^\prime,y^\prime)\mathrm{d}A
					\end{equation*} }
					\only<2>{\begin{equation*}
						r_f(x,y)=\int_{A_{px}}f(x-x^\prime,y-y^\prime)r(x^\prime,y^\prime)\mathrm{d}A
					\end{equation*} }
					\only<3>{\begin{equation*}
						r_f(x,y)\approx\frac{\norm{A_{px}}}{n}\sum_{i=1}^{n}f(x-x_i,y-y_i)r(x_i,y_i)
					\end{equation*} }
				}
				\only<4>{
					\scriptsize
					\begin{tabularx}{\linewidth}{X}
					\toprule
					\textbf{Path Tracing}\label{pathTracing} \\
					\midrule
					\begin{algorithmic}
						\Inputt scena $\sigma$, risoluzione richiesta $(w,h)$, camera $c$
						\Outputt funzione immagine filtrata $r_f(x,y)$
						\State $r_f$ \gets Array($w$,$h$);
						\For{\textbf{each} pixel $(x,y)$ \textbf{in} $r_f$}
							\State ss \gets GenerateSamplesWithinPixel(x,y);
							\For{\textbf{each} sample ($\vec{s}$) \textbf{in} ss.positions}
								\State $\rho$ \gets CastRay$\left(c.\vec{p},\tfrac{\vec{s}-c.\vec{p}}{\norm{\vec{s}-c.\vec{p}}}\right)$;
								\State ss.contributions($\vec{s}$)\gets SampleRadiance($\sigma$,$\rho$, depth$=0$);
							\EndFor
							\State $r_f(x,y)$\gets Aggregate(ss.contributions);
						\EndFor
					\end{algorithmic}\\
					\bottomrule
					\end{tabularx}
				}
			\end{column}
		\end{columns}
	\end{frame}

	\begin{frame}{Indice}
		\tableofcontents[hideallsubsections]
	\end{frame}

	\AtBeginSection[ ]
	{
	\begin{frame}{Indice: \currentsectionname}
		\tableofcontents[currentsection,hideallsubsections]
	\end{frame}
	}

	\section{Monte Carlo Integration}
	\begin{frame}{Preliminari}
		\only<1>{
			\begin{altDescription}{probDefs}
				\item[Probabilit\`a $\Pr(A)$] Funzione definita su un dominio $\Sigma$ avente immagine $[0,1]$.
				\item[Variabile Casuale $X$] Formalizzazione matematica di quantit\`a numerica dipendente da eventi aleatori.
			\end{altDescription}
			\hbox{\hfill}
			Caratterizzata univocamente da
			\begin{altDescription}{PDF-CDF}
				\item[CDF] %Funzione associante ad una variabile casuale $X$ la probabilit\`a di restituire un'osservazione $\leq$ ad un campione $X$
					$\displaystyle P(x)=\Pr(X\leq x)$
				\item[PDF] %Funzione associante ad una variabile casuale $X$ la "probabilit\`a per unit\`a di lunghezza" associata ad un campione $x$.
					%Intuitivamente, probabilit\`a per $X\in[x,x+\mathrm{d}x]$
					$\displaystyle p(x)=\frac{\mathrm{d}}{\mathrm{d}x}P(x)$
			\end{altDescription}
		}
		\only<2>{
			\begin{altDescription}{d}
				\item[Stimatore $\tilde{F}_n$] funzione di una collezioni di variabili aleatorie $X_i$, $n$ campioni, mappati ad una 
					stima di uno stimando $F$
			\end{altDescription}
			\begin{align*}
			\begin{array}{lrl}
				\mathsf{Aspettazione: }& E[f(X)]&=\displaystyle\int_{\mathcal{D}}f(x)p(x)\mathrm{d}x \\
				\mathsf{Varianza: }& V[f(X)]&=E\left[(f(X)-E[f(X)])^2\right] \\
				\mathsf{Bias: }& \beta&=E\left[\tilde{F}_n(X_1,\ldots,X_n)\right]-F \\
				\mathsf{Efficienza: }& \epsilon\left[\tilde{F}_n\right]&=\displaystyle\frac{1}{T\left[\tilde{F}_n\right]V\left[\tilde{F}_n\right]} \\
				\mathsf{MSE: }& \mathit{MSE}\left[\tilde{F}_n\right]&=E\left[(\tilde{F}_n-F)^2\right]=V\left[\tilde{F}_n\right]+\beta^2
			\end{array}
			\end{align*}
		}
	\end{frame}
	\begin{frame}{Integrazione di Monte Carlo}
		\only<1-2>{Stimatore Unbiased per l'Integrazione Monte Carlo}
		\only<1>{\begin{equation*}
			\tilde{F}_n=\norm{\mathcal{D}}\tilde{E}_n[f(X)]=\frac{1}{n}\sum_{i=1}^n\frac{f(X_i)}{p(x)}
		\end{equation*}
			Il campionamento secondo PDF $p(x)$ permette di scegliere accuratamente i campioni pi\`u significativi (\textit{Importance Sampling})
		}
		\only<2->{\begin{equation*}
			\tilde{F}_n=\norm{\mathcal{D}}\tilde{E}_n[f(X)]=\frac{\norm{\mathcal{D}}}{n}\sum_{i=1}^nf(X_i)
		\end{equation*}}
		\only<2>{(Campionamento $\mathcal{U}(\mathcal{D})$)}
		\only<3>{
			Per $n$ campioni, la deviazione standard diminuisce di $\sqrt{n}$
			\begin{equation*}
				\sigma[\tilde{F}_n]=
					\sqrt{V[\tilde{F}_n]}=\sqrt{\frac{\norm{\mathcal{D}}^2}{n}V[f(X)]}=\frac{\norm{\mathcal{D}}}{n^{\frac{1}{2}}}\sigma[f(X)]
			\end{equation*}
		}
	\end{frame}
	\begin{frame}[t,fragile]
		\frametitle{Esempio: Integrale monodimensionale}
		\only<1>{\begin{columns}
			\begin{column}{0.35\textwidth}
				\begin{scaletikzpicturetowidth}{\linewidth}\begin{tikzpicture}[scale=\tikzscale]
					\begin{axis} [
						axis lines = left,
						xlabel={$x$}, ylabel={$f(x)$},
						every axis y label/.style={at=(current axis.above origin),anchor=south east},
						every axis x label/.style={at=(current axis.right of origin),anchor=north west},
						ymin=0, ymax=4,
						xmin=0, xmax=11,
						declare function={
							s(\x)=sqrt(((pi^2)/10)*\x);
							func(\x)=sin(s(\x)*180/pi)*e^s(\x)/s(\x);
						},
					]
						\addplot[color=black, domain=0.001:11, domain y=0:4, samples=50,]{func(x)};
					\end{axis}
				\end{tikzpicture}\end{scaletikzpicturetowidth}
				{\small\begin{equation*}
					\int_0^{10}\frac{\sin\left(\sqrt{\frac{\pi^{2}}{10}x}\right)}{\sqrt{\frac{\pi^{2}}{10}x}}e^{\sqrt{\frac{\pi^{2}}{10}x}}\mathrm{d}x
				\end{equation*}}
			\end{column}
			\begin{column}{0.65\textwidth}
				\`e possibile giungere alla soluzione analitica
				\begin{equation*}
					F=\frac{10e^\pi+10}{\pi^2}\approx 24.45963551499059
				\end{equation*}
				Possiamo applicare il Metodo di Monte Carlo
				\begin{equation*}
					\tilde{F}_n=\norm{\mathcal{D}}\tilde{E}_n[f(X)]=\frac{10}{n}\sum_{i=1}^nf(X_i)
				\end{equation*}
				Campionando uniformemente $[0,10]$
			\end{column}
		\end{columns}}
		\only<2-3>{\begin{figure}[ht]
			\scriptsize	
			\only<2>{\begin{subfigure}[t]{0.45\textwidth}
				\begin{scaletikzpicturetowidth}{\textwidth}\begin{tikzpicture}[scale=\tikzscale]
					\begin{axis} [
						axis lines = left,
						xlabel={$n$}, ylabel=$\tilde{F}_n-F$,
						every axis y label/.style={at=(current axis.above origin),anchor=south east},
						every axis x label/.style={at=(current axis.right of origin),anchor=north west},
						ymax=2, ymin=-2,
					]
						\addplot [black,] table[ignore chars={(,)},col sep=comma] {../assets/chapter6_error_data.dat};
					\end{axis}
				\end{tikzpicture}\end{scaletikzpicturetowidth}
				\caption{Funzione Errore per $\tilde{F}_n$}
			\end{subfigure}
			\begin{subfigure}[t]{0.45\textwidth}
				\begin{scaletikzpicturetowidth}{\textwidth}\begin{tikzpicture}[scale=\tikzscale]
					\begin{axis} [
						axis lines = left,
						xlabel={$n$}, ylabel={$\epsilon\left[\tilde{F}_n\right]$},
						every axis y label/.style={at=(current axis.above origin),anchor=south east},
						every axis x label/.style={at=(current axis.right of origin),anchor=north west},
						ymax=5, ymin=0,
					]
						\addplot [black,] table[ignore chars={(,)},col sep=comma] {../assets/chapter6_efficiency.dat};
					\end{axis}
				\end{tikzpicture}\end{scaletikzpicturetowidth}
				\caption{Funzione efficienza $\epsilon\left[\tilde{F}_n\right]$}
			\end{subfigure}}
			\only<3>{\begin{subfigure}[t]{0.45\textwidth}
				\begin{scaletikzpicturetowidth}{\textwidth}\begin{tikzpicture}[scale=\tikzscale]
					\begin{axis} [
						axis lines = left,
						xlabel={$n$}, ylabel={$MSE\left[\tilde{F}_n\right]$},
						every axis y label/.style={at=(current axis.above origin),anchor=south east},
						every axis x label/.style={at=(current axis.right of origin),anchor=north west},
						ymax=5, ymin=0,
					]
						\addplot [black,] table[ignore chars={(,)},col sep=comma] {../assets/chapter6_mse_data.dat};
					\end{axis}
				\end{tikzpicture}\end{scaletikzpicturetowidth}
				\caption{$MSE[\tilde{F}_n]$ dello stimatore unbiased}
			\end{subfigure}
			\begin{subfigure}[t]{0.45\textwidth}
				\begin{scaletikzpicturetowidth}{\textwidth}\begin{tikzpicture}[scale=\tikzscale]
					\begin{axis} [
						axis lines = left,
						xlabel={$n$}, ylabel=$\tilde{F}_n^{\beta}-F$,
						every axis y label/.style={at=(current axis.above origin),anchor=south east},
						every axis x label/.style={at=(current axis.right of origin),anchor=north west},
						ymin=-2, ymax=0,
					]
						\addplot [black,] table[ignore chars={(,)},col sep=comma] {../assets/chapter6_biased_estimator_error_data.dat};
					\end{axis}
				\end{tikzpicture}\end{scaletikzpicturetowidth}
				\caption{Funzione errore per stimatore biased $\tilde{F}_n^{\beta}=\tfrac{1}{2}\operatorname{max}\{f(X_1),\ldots\}$}
			\end{subfigure}}
		\end{figure}}
	\end{frame}

	\begin{frame}\frametitle{Metodi di Riduzione della Varianza}
		\begin{altDescription}{variance}
			\item[Stratified Sampling] Suddivisione del dominio di integrazione in $n$ regioni chiamate \textit{strata}
			\item[Importance Sampling] Concentra i punti campionati per la stima dell'integrale utilizzando PDF
				proporzionale alla funzione integranda, per catturarne pi\`u rapidamente i contributi pi\`u significativi
			\item[Russian Roulette] Salta la valutazione di campioni che contribuiscono poco
			\item[Splitting] Aumenta il numero di campioni in alcune dimensioni di un integrale multidimensionale
		\end{altDescription}
	\end{frame}
	
	\section{Rendering Fundamentals}

	\begin{frame}
		\frametitle{Radiometria}
		\only<1>{
			Modello matematico per studiare e misurare la propagazione delle radiazioni elettromagnetiche con la sola ottica geometrica.
			\begin{itemize}
				\item[\bullet] \textsf{Linearit\`a}
				\item[\bullet] \textsf{Conservazione dell'energia}
				\item[\bullet] \textsf{No polarizzazione}
				\item[\bullet] \textsf{No fluorescenza o fosforescenza}
				\item[\bullet] \textsf{A regime}
			\end{itemize}
		}
		\only<2>{\begin{table}[tb]
			\betweensmallandscript
			\begin{tabularx}{\linewidth}{>{\textsf}l >{$}p{\widthof{Simbolo}}<{$} p{\widthof{$\si{W/(sr.m^2)}$}} r@{}l}
				\toprule
				\textrm{Quantit\`a} & \text{Simbolo} & unit\`a S.I. & \multicolumn{2}{c}{Formula} \\
				\midrule
				{Energia Radiante}	 & Q_e			& $\si{J}$	& $\displaystyle Q_{e,\lambda}=$&$\displaystyle\frac{\partial Q_e}{\partial\lambda}$ \\
				{Flusso Radiante}	 & \Phi_e		& $\si{W}$	& $\displaystyle\Phi_e=$		&$\displaystyle\frac{\partial Q_e}{\partial t}$ \\
				{Intensita Radiante} & I_{e,\Omega} &$\si{W/sr}$& $\displaystyle I_{e,\Omega}=$ &$\displaystyle\frac{\partial\Phi_e}{\partial\Omega}$\\
				{Irradianza}		 & E_e			& \multirow{3}{*}{$\si{W/m^2}$}  
													& \multirow{3}{*}{$\displaystyle E_e | J_e | M_e=$} 
													& \multirow{3}{*}{$\displaystyle\frac{\partial\Phi_e}{\partial A}$} \\
				{Emittanza Radiante} & M_e 			&  & &\\
				{Radiosita}			 & J_e 			&  & &\\
				{Radianza}			 & L_{e,\Omega} & $\si{W/(sr.m^2)}$& $\displaystyle L_{e,\Omega}=$	
																	   & $\displaystyle\frac{\partial^2\Phi_e}{\partial\Omega\partial(A\cos\theta)}$ \\
				\bottomrule
			\end{tabularx}
			\normalsize
			\caption{nomenclatura e misure delle quantit\`a radiometriche per noi rilevanti}
			\label{chapter1:radiometric-quantities}
		\end{table}}
	\end{frame}

	%\begin{frame}[t,fragile]\frametitle{Fotometria e Colorimetria}
	%	\begin{columns}[T]
	%		\begin{column}{0.35\textwidth}
	%			\begin{scaletikzpicturetowidth}{\linewidth}\begin{tikzpicture}[scale=\tikzscale]
	%				\begin{axis} [
	%					axis lines = left,
	%					xlabel={$\lambda\,[\si{nm}]$}, ylabel={$\bar{x}(\lambda),\bar{y}(\lambda),\bar{z}(\lambda)$},
	%					every axis y label/.style={at=(current axis.above origin),anchor=south east},
	%					every axis x label/.style={at=(current axis.right of origin),anchor=north west},
	%					xlabel absolute,
	%					ymin=0, ymax=2,
	%					xmin=360, xmax=830,
	%				]
	%					\addplot [red,] table[x=L, y=X,col sep=comma] {./assets/CIE_xyz_1931_2deg.csv};
	%					\addplot [green,] table[x=L, y=Y,col sep=comma] {./assets/CIE_xyz_1931_2deg.csv};
	%					\addplot [blue,] table[x=L, y=Z,col sep=comma] {./assets/CIE_xyz_1931_2deg.csv};
	%				\end{axis}
	%			\end{tikzpicture}\end{scaletikzpicturetowidth}
	%		\end{column}
	%		\begin{column}{0.65\textwidth}
	%			\only<1>{Ciascuna grandezza radiometrica $X_e$ pu\`o essere scomposta nella sua controparte spettrale
	%			\begin{equation*}
	%				X_{e,\lambda}(\lambda) = \tfrac{\mathrm{d}X_e}{\mathrm{d}\lambda}
	%			\end{equation*}}
	%			\only<2>{\textsf{Fotometria:} modulazione delle grandezze radiometriche secondo la percezione umana, tramite la Funzione di 
	%			Efficacia Luminosa Spettrale Fotopica $V(\lambda)$ e l'Efficacia Luminosa Fotopica $K=683.002\,\si{lm/W}$}
	%			\only<3>{\textsf{Colorimetria:} modulazione delle grandezze radiometriche secondo la percezione umana, scomponendole secondo 
	%			distribuzioni spettrali costituenti la base di uno spazio vettoriale (Color Space), $\bar{x},\bar{y},\bar{z}$
	%			}
	%			\only<4>{\textsf{sRGB$\rightleftarrows$Radianza Spettrale}:
	%			\begin{altDescription}{spec}
	%				\item[Spettro$\to$sRGB] Utilizzo di una LUT precomputata per ottenere un colore XYZ. Dopodich\`e Trasformare tale colore in sRGB
	%				\item[sRGB$\to$Spettro] Applicare trasformazione inversa per ottenere un colore XYZ. Dopodich\`e somma pesata delle tre CMFs
	%			\end{altDescription}}
	%		\end{column}
	%	\end{columns}
	%	\only<2->{\vbox{\vfill}}
	%	\only<2>{\small\begin{equation*}
	%		L_v = K\int_{360\,\si{nm}}^{830\,\si{nm}}L_{e,\lambda}(\lambda)V(\lambda)\mathrm{d}\lambda\;[\si{nit}=\si{cd/m^2}]
	%	\end{equation*}
	%	per definizione, $V(\lambda)=\bar{y}(\lambda)$
	%	}
	%	\only<3>{\small\begin{equation*}
	%		X|Y|Z = \frac{1}{\int_\Lambda L_{e,\Omega,\lambda}(\lambda)\bar{y}(\lambda)\mathrm{d}\lambda}
	%			\int_\Lambda S(\lambda)L_{e,\Omega,\lambda}(\lambda)\bar{x}|\bar{y}|\bar{z}(\lambda)\mathrm{d}\lambda
	%	\end{equation*}}
	%	\only<4>{\tiny\begin{align*}
	%		\begin{array}{cc}
	%			C_{linear} = \left\{\begin{alignedat}{2}
	%				&\frac{C_{srgb}}{12.92}, &C_{srgb}\leq 0.04045\\
	%				&\left(\frac{C_{srgb}+0.055}{1.055}\right)^{2.4},\;&C_{srgb}> 0.04045
	%			\end{alignedat}\right.
	%			&
	%			\begin{bmatrix}
	%				X \\ Y \\ Z
	%			\end{bmatrix}=
	%			\begin{bmatrix}
	%				0.4124 & 0.3576 & 0.1805 \\
	%				0.2126 & 0.7152 & 0.0722 \\
	%				0.0193 & 0.1192 & 0.9505 
	%			\end{bmatrix}
	%			\begin{bmatrix}
	%				R_{linear} \\ G_{linear} \\ B_{linear}
	%			\end{bmatrix}
	%		\end{array}
	%	\end{align*}}
	%\end{frame}

	\begin{frame}{Rendering Equation}
		\only<1>{\begin{figure}[tb]
			\fontsize{8pt}{9.6pt}
			\centering
			\begin{subfigure}[b]{0.3\textwidth}
				\begin{scaletikzpicturetowidth}{\linewidth}\begin{tikzpicture}[scale=\tikzscale]
					\def\ang{30}
					\def\projang{15}
					\def\projdist{0.2cm}
					\def\ray{1.7cm}
					\coordinate (O) at (0,0); % origin
					\coordinate (OR) at (canvas polar cs:angle=90-\ang, radius=\ray-\projdist); % reflected direction
					\node[at={(0,0)}, hemi]{};
					\draw[light beam] (O) -- (OR) node [anchor=north, right=0.1] {$\hat{\omega}_r$};
					\draw pic["",draw=black,angle radius=20,angle eccentricity=1.3]{angle = OR--O--N}node[anchor=north,above=23,right=0.05]{$\theta_r$};
				\end{tikzpicture}\end{scaletikzpicturetowidth}
				\caption{Specular BRDF}
			\end{subfigure}\hspace*{\fill}
			\begin{subfigure}[b]{0.3\textwidth}
				\begin{scaletikzpicturetowidth}{\linewidth}\begin{tikzpicture}[scale=\tikzscale]
					\coordinate (O) at (0,0);
					\node[at={(O)}, hemi]{};

					\begin{scope}[scale=0.4, every node/.style={scale=0.4}, rotate=-40]
						\def\r{1} % r is the radius of the cone base
						\def\R{1.042*\r}   % R is the radius of the sphere
						\pgfmathsetmacro\h{sqrt(\R*\R-\r*\r)} 
						\pgfmathsetmacro\a{asin(\r/\R)}
						\pgfmathsetmacro\H{\r*tan(\a)}
						
						\coordinate (Apex) at (0,0);
						\coordinate (BaseLeft) at ($(Apex) + (-\r,\H)$);%(-\r,0);
						\coordinate (SphereCenter) at ($(Apex) + (0,\h+\H)$);
						\coordinate (BaseRight) at ($(Apex) + (\r,\H)$);%(\r,0);
						
						% Wireframe
						\draw[fill=myblue!70] (BaseRight) arc (-90+\a:270-\a:\R);
						\draw[fill=myblue!70] (BaseLeft) -- (Apex) -- (BaseRight);
						\draw[fill=myblue!70] (BaseRight) arc (0:-180:\r cm and 0.2*\r cm);
						\draw[dashed, fill=myblue!70] (BaseRight) arc (0:180:\r cm and 0.2*\r cm);
					\end{scope}

					% low probability hemisphere
					\draw[fill=myblue, opacity=.7] 
						($(R) - (1.5,0)$) arc (0:180:0.5cm) arc (180:360:0.5cm and 0.15cm);

					% outline of the reflection hemisphere
					\draw ($(R) - (1.5,0)$) arc (0:180:0.5cm) arc (180:360:0.5cm and 0.15cm); 
					\draw [dashed] ($(R) - (1.5,0)$) arc (0:180:0.5cm and 0.15cm);

				\end{tikzpicture}\end{scaletikzpicturetowidth}

				\caption{Glossy BRDF}
			\end{subfigure}\hspace*{\fill}
			\begin{subfigure}[b]{0.3\textwidth}
				\begin{scaletikzpicturetowidth}{\linewidth}\begin{tikzpicture}[scale=\tikzscale]
					\coordinate (R) at (2,0); % point to the right
					\node[at={(0,0)}, hemi]{};

					% diffuse reflection ball
					\draw[fill=myblue, opacity=.7] 
						($(R) - (1,0)$) arc (0:180:1cm) arc (180:360:1cm and 0.3cm);
						
					% outline of the reflection hemisphere
					\draw ($(R) - (1,0)$) arc (0:180:1cm) arc (180:360:1cm and 0.3cm); 
					\draw [dashed] ($(R) - (1,0)$) arc (0:180:1cm and 0.3cm);
				\end{tikzpicture}\end{scaletikzpicturetowidth}
				\caption{Diffuse BRDF}
			\end{subfigure}
		\end{figure}}
		\only<1-2>{\begin{equation*}
			L_o(\vec{p},\hat{\omega}_o) = L_e(\vec{p},\hat{\omega}_o) + \int_{\mathcal{S}^2}%
				L_o(t(\vec{p}, \hat{\omega}_i),-\hat{\omega}_i)f_s(\vec{p},\hat{\omega}_o,\hat{\omega}_i)%
				\vert\langle\hat{n},\hat{\omega}_i\rangle\vert\mathrm{d}\hat{\omega}_i
		\end{equation*}}
		\only<2>{
			\begin{itemize}
				\item[\bullet]$\vec{p},\hat{\omega}_i,\hat{\omega}_o$ risp. punto superficie considerato, direzione incidente considerata, 
					direzione uscente
				\item[\bullet] $t(\vec{p},\hat{\omega}_i)$ \textit{Ray Tracing Function}: dati punto di partenza e direzione, restituisce un punto
					$\vec{q}$, prima intersezione incontrata
				\item[\bullet] $\mathcal{S}^2$ sfera unitaria di centro $\vec{p}$
				\item[\bullet] $\int_{\mathcal{S}^2}$ Integrale in tutte le direzioni $\hat{\omega}_i$
			\end{itemize}
		}
	\end{frame}

	\begin{frame}[t,fragile]\frametitle{BSDF}
		\only<1>{
			Densit\`a di Distribuzione\footnote{Non normalizzata, con integrale su emisfera proiettata pari all'albedo della superficie} di radianza 
			incidente riflessa (BRDF $f_r$) o trasmessa (BTDF $f_t$) in una data direzione
			\begin{equation*}
				f_s(\vec{p},\hat{\omega}_o,\hat{\omega}_i) = f_r(\vec{p},\hat{\omega}_o,\hat{\omega}_i) + f_t(\vec{p},\hat{\omega}_o,\hat{\omega}_i)
			\end{equation*}
			dove
			\begin{align*}
				f_r(\vec{p},\hat{\omega}_o,\hat{\omega}_i) 
					&= 0,\;\;\mathrm{se}\,\langle\hat{n},\hat{\omega}_o\rangle\langle\hat{n},\hat{\omega}_i\rangle\leq0\\
				f_t(\vec{p},\hat{\omega}_o,\hat{\omega}_i) 
					&= 0,\;\;\mathrm{se}\,\langle\hat{n},\hat{\omega}_o\rangle\langle\hat{n},\hat{\omega}_i\rangle\geq0
			\end{align*}
		}
		\only<2-3>{\begin{columns}[T]
			\only<2>{\begin{column}{0.5\textwidth}
				\textsf{Lambertian BRDF:}
				\begin{center}\scalebox{0.6}{\begin{scaletikzpicturetowidth}{\linewidth}\begin{tikzpicture}[scale=\tikzscale]
					\coordinate (R) at (2,0); % point to the right
					\node[at={(0,0)}, hemi]{};

					% diffuse reflection ball
					\draw[fill=myblue, opacity=.7] 
						($(R) - (1,0)$) arc (0:180:1cm) arc (180:360:1cm and 0.3cm);
						
					% outline of the reflection hemisphere
					\draw ($(R) - (1,0)$) arc (0:180:1cm) arc (180:360:1cm and 0.3cm); 
					\draw [dashed] ($(R) - (1,0)$) arc (0:180:1cm and 0.3cm);
				\end{tikzpicture}\end{scaletikzpicturetowidth}}\end{center}
				\begin{equation*}
					f_r(\vec{p}) = \frac{\rho(\vec{p})}{\pi}
				\end{equation*}
			\end{column}}
			\begin{column}{0.5\textwidth}
				\textsf{Perfectly Specular BRDF:}
				\begin{center}\scalebox{0.6}{\begin{scaletikzpicturetowidth}{\linewidth}\begin{tikzpicture}[scale=\tikzscale]
					\def\ang{30}
					\def\projang{15}
					\def\projdist{0.2cm}
					\def\ray{1.7cm}
					\coordinate (O) at (0,0); % origin
					\coordinate (OR) at (canvas polar cs:angle=90-\ang, radius=\ray-\projdist); % reflected direction
					\node[at={(0,0)}, hemi]{};
					\draw[light beam] (O) -- (OR) node [anchor=north, right=0.1] {$\hat{\omega}_r$};
					\draw pic["",draw=black,angle radius=20,angle eccentricity=1.3]{angle= OR--O--N}node[anchor=north,above=23,right=0.05]{$\theta_r$};
				\end{tikzpicture}\end{scaletikzpicturetowidth}}\end{center}
				\begin{equation*}
					\only<2>{\hspace*{-1em}}f_r(\vec{p},\hat{\omega}_o,\hat{\omega}_i)=F_r(\langle\hat{n},\hat{\omega}_o\rangle)%
						\frac{\delta(\hat{\omega}_i-\hat{\omega}_r)}{\vert\langle\hat{n},\hat{\omega}_i\rangle\vert}
				\end{equation*}
			\end{column}
			\only<3>{\begin{column}{0.5\textwidth}
				\begin{align*}
					F_r(\mu) &\approx F_0+(1-F_0)(1-\mu)^5 \\
					F_0 &= \frac{(\eta-1)^2+\kappa^2}{(\eta+1)^2+\kappa^2} \stackrel{\kappa\approx0}{=} \left(\frac{\eta-1}{\eta+1}\right)^2\\
					\hat{\omega}_r&=2\langle\hat{n},\hat{\omega}_i\rangle - \hat{\omega}_i
				\end{align*}
			\end{column}}
		\end{columns}}
	\end{frame}

	\section{Campionamento e Ricostruzione}
	\begin{frame}\frametitle{Campionamento}
		\only<1>{
			Affich\`e si sfruttino al meglio i campioni estratti per lo Stimatore di Monte Carlo
			\begin{itemize}[topsep=0.1pt,noitemsep]
				\item[\bullet]Riduzione della Varianza per lo stimatore di Monte Carlo tramite scelta accurata della distribuzione dei campioni...
				\item[\bullet]...Con conseguente bisogno di algoritmi capaci di estrarre osservazioni da una distribuzione
					arbitraria, dati campioni distribuiti uniformemente in $\mathcal{U}(0,1)$
			\end{itemize}
		}
		\only<2>{
			\begin{columns}
			\begin{column}{0.35\textwidth}
				\includegraphics[width=\linewidth]{../assets/intro_path_tracing_online_tech_tips.png}
				\small Stima della funzione immagine $r_f(x,y)$ attraverso Path Tracing.
			\end{column}
			\begin{column}{0.65\textwidth}
			In un sistema di rendering,
			\begin{itemize}[topsep=0.1pt,noitemsep]
				\item[\bullet] Ciascun pixel \`e campionato generando raggi e accumulando radianza in ciascuno di essi
				\item[\bullet] Ogni superficie intersecata da ciascun raggio \`e campionata per generare un nuovo raggio
			\end{itemize}
			\end{column}
			\end{columns}
		}
		\only<3>{
			\begin{table}[t]
			\begin{tabularx}{\textwidth}{>{\textsf}cc}
				\toprule	
				{Variance Reduction} & {PDF Sampling} \\
				\midrule
				Stratified Sampling & Inverse Transform Sampling \\
				Importance Sampling & Acceptance-Rejection Sampling \\
				Russian Roulette & Metropolis-Hastings Sampling \\
				Splitting & \\
				\bottomrule
			\end{tabularx}
			\end{table}
			Inverse Transform Sampling Lambertian BRDF ($\xi_\theta,\xi_\varphi\sim\mathcal{U}(0,1)$)
			\begin{align*}
				\varphi_i&=2\pi\xi_\varphi \\
				\theta_i&=\arccos\left(\sqrt{1-\xi_\theta}\right)
			\end{align*}
		}
	\end{frame}
	
	\begin{frame}\frametitle{Filtro di Ricostruzione}
		\begin{figure}[tb]
			\centering 
			\includegraphics[width=0.6\linewidth]{../assets/chapter5_reconstruction_box.png}
		\end{figure}
		Ricostruzione della funzione immagine dai suoi campioni.\\
		\textsf{Box Filter}: considera solamente campioni all'interno del pixel considerato, aggregati come spiegato nell'Introduzione
	\end{frame}

	%\section{Path Tracing}
	%\begin{frame}\frametitle{Rendering Equation, "Path Form"}
	%\end{frame}

	%\begin{frame}\frametitle{Path Tracing, Espressione}
	%\end{frame}
	
	\section{Simulazione}
	\begin{frame}\frametitle{Simulazione: Cornell Box}
		\begin{columns}
			\begin{column}{0.6\textwidth}
				Path Tracer semplificato, supportante
				\begin{itemize}[topsep=0.1pt,noitemsep]
				\item[\bullet] Tre tipi di BRDF, in particolare Lambertiana, Speculare opaca o traslucente
				\item[\bullet] \textsf{Russian Roulette}, applicata per terminare la valutazione di una path
				\item[\bullet] \textsf{Stratified Sampling}, dividendo ciascun pixel in 4 caselle (strata)
				\item[\bullet] Ciascun sample in ciascun strata è scelto casualmente con PDF triangolare
				\end{itemize}
			\end{column}
			\begin{column}{0.4\textwidth}
				\begin{figure}[t]
					\centering
					\includegraphics[width=\textwidth]{../assets/appendixD_result_25k.png}
				\end{figure}
				Scena \textsf{Cornell Box}.
			\end{column}
		\end{columns}
	\end{frame}

	\begin{frame}{Risultati Simulazione}
		\showslide{1}{8}{40}
		\showslide{3}{200}{1000}
		\showslide{5}{5k}{25k}
		\only<2>{
			\begin{table}[tb]
				\small
				\begin{tabularx}{\linewidth}{XXX}
					\toprule
					\multirow{2}{*}{Componente analizzata} & \multicolumn{2}{>{\hsize=\dimexpr2\hsize- + 2\tabcolsep\relax}c}{Varianza trovata} \\
					& 8 campioni per stratum & 40 campioni per stratum \\
					\midrule
					Red Channel					& 0.07344142 & 0.05132978 \\
					Green Channel				& 0.06190188 & 0.04714374 \\
					Blue Channel				& 0.07339846 & 0.05175589 \\
					Luminance\footnotemark{}	& 0.06380390 & 0.04621717 \\
					\bottomrule
				\end{tabularx}
				\caption{Tabella delle varianze per la resa con 8 campioni per stratum e 40 campioni per stratum}
			\end{table}
			\footnotetext{Pi\`u precisamente, Luma channel del color space ITU-R BT.709}
		}
		\only<4>{
			\begin{table}[tb]
				\small
				\begin{tabularx}{\linewidth}{XXX}
					\toprule
					\multirow{2}{*}{Componente analizzata} & \multicolumn{2}{>{\hsize=\dimexpr2\hsize- + 2\tabcolsep\relax}c}{Varianza trovata} \\
					& 200 campioni per stratum & 1000 campioni per stratum \\
					\midrule
					Red Channel					& 0.02482136 & 0.02023584 \\
					Green Channel				& 0.02453323 & 0.02019825 \\
					Blue Channel				& 0.02517796 & 0.02031221 \\
					Luminance					& 0.02276842 & 0.01840096 \\
					\bottomrule
				\end{tabularx}
				\caption{Tabella delle varianze per la resa con 200 campioni per stratum e 1000 campioni per stratum}
			\end{table}
		}
		\only<6>{
			\begin{table}[tb]
				\small
				\begin{tabularx}{\linewidth}{XXX}
					\toprule
					\multirow{2}{*}{Componente analizzata} & \multicolumn{2}{>{\hsize=\dimexpr2\hsize- + 2\tabcolsep\relax}c}{Varianza trovata} \\
					& 5000 campioni per stratum & 25000 campioni per stratum \\
					\midrule
					Red Channel					& 0.01904086 & 0.01885027 \\
					Green Channel				& 0.01936775 & 0.01919554 \\
					Blue Channel				& 0.01930308 & 0.01911878 \\
					Luminance					& 0.01754279 & 0.01737635 \\
					\bottomrule
				\end{tabularx}
				\caption{Tabella delle varianze per la resa con 5000 campioni per stratum e 25000 campioni per stratum}
			\end{table}
		}
	\end{frame}

\end{document}

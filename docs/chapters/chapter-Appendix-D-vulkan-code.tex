\chapter{Codice}
\section{Owen Scrambling di punti della Sequenza di Halton}\label{appendixD:owenScrambling}
\begin{minted}[tabsize=2,obeytabs]{C++}
// baseIndex <- base dell'Inverso Radicale
// a         <- cifre dell'inverso radicale d1...dm
// hash      <- valore casuale
Float OvenScrambledRadicalInverse(int baseIndex, uint64_t a, 
                                  uint32_t hash)
{
	int base = Primes[baseIndex];
	Float invBase = 1.f / base, invBaseM = 1;

	// bits Scrambled finora
	uint64_t reversedDigits = 0;
	int digitIndex = 0;

	// finche non si esauriscono le cifre di a da 
	// mettere nel floating point
	while (1 - invBaseM < 1)
	{
		// estrai il valore della <digitIndex> cifra.
		// Equivalente a un right shift del numero
		// in base <base>
		uint64_t next = a / base;
		int digitValue = a - next * base;

		// calcola un seed unico per permutare casualmente il valore 
		// estratto nota che tale seed dipende da tutti i valori 
		// precedentemente permutati
		uint32_t digitHash = MixBits(hash ^ reversedDigits);

		// seleziona la <digitValue>-esima permutazione 
		// pseudocasuale basata sul seed <digitHash>, 
		// di <base> numeri casuali
		digitValue = PermutationElement(digitValue, base, digitHash);

		// aggiungi la cifra a quelle calcolate finora
		reversedDigits = reversedDigits * base + digitValue;

		invBaseM *= invBase;
		++digitIndex;
		a = next;
	}
	return std::min(invBaseM * reversedDigits, OneMinusEpsilon);
}
\end{minted}
\section{Acceptance-Rejection Sampling di un disco unitario}\label{appendixD:rejectionSampling}
\begin{minted}[tabsize=2,obeytabs]{python}
import numpy as np

def target_pdf(x, y):
    return (4 / np.pi) if (x**2 + y**2 <= 1 and x >= 0 and y >= 0) 
	else 0

def instrumental_pdf(x, y):
    return 1 if (0 <= x <= 1 and 0 <= y <= 1) else 0

def acceptance_probability(x, y, b):
    return 

def rejection_sampling(num_samples):
    samples = []
    accepted_samples = 0
	 b = 4 / np.pi

    while accepted_samples < num_samples:
        x = np.random.uniform(0, 1)
        y = np.random.uniform(0, 1)
        if np.random.uniform(0, 1)*b*target_pdf(x, y) 
		<= instrumental_pdf(x, y):
            samples.append((x, y))
            accepted_samples += 1

    return samples

def alt_rejection_sampling(num_samples):
	samples = []
	accepted_samples = 0
	attempts = 0
    while accepted_samples < num_samples:
		attempts++
        x = np.random.uniform(0, 1)
        y = np.random.uniform(0, 1)
        if x**2+y**2 <= 1:
            samples.append((x, y))
            accepted_samples += 1

	pi_estimate = accepted_samples / attempts
	return (samples, pi_estimate)
\end{minted}
\section{Metropolis-Hastings Sampling per campionare $\mathcal{N}(0,1)$}\label{appendixD:metropolisHastings}
\begin{minted}[tabsize=2,obeytabs]{python}
import math
import numpy as np
from numpy.random import default_rng

_rng : Generator = default_rng()

def normPdf(value, mean = 0.0, stddev = 1.0):
    rSqrt2Pi = 0.39894228
    rStddev = 1 / stddev
    return rSqrt2Pi * rStddev * math.exp(-0.5*((value - mean)* rStddev)**2)

def metropolisHastings(samples, N):
    for i in np.arange(N-1):
        u = _rng.uniform()
        proposedSample = _rng.normal(samples[i], 0.05)

		acceptance = 
			np.clip(min(1, normPdf(proposedSample) / normPdf(samples[i])), 0., 1.)
        if u < acceptance:
            samples[i+1] = proposedSample
        else:
            samples[i+1] = samples[i]

    return
\end{minted}
\section{Monte Carlo Integration per un integrale semplice}\label{appendixD:MC}
Ripreso dal codice della repository del progetto:
\begin{minted}[tabsize=2,obeytabs]{python}
import math
import numpy as np
import numpy.typing as npt
import sys
from numpy.random import default_rng
from typing import Generator, Annotated, Callable
from annotated_types import Gt

_rng : Generator = default_rng()

def integrand(x : float):
    arg : float = math.pi*math.sqrt(x*0.1)
    return (math.sin(arg)/arg)*math.e**arg;

# suppone il calcolo dell'integrale definito e' nell'intervallo [0,10]
def trueIntegralValue():
    return (10*math.e**math.pi+10)/(math.pi**2)

def monteCarloIntegration(integrandFunction : Callable[[float], float], 
                          domainSize : float, nSamples : Annotated[int, Gt(0)]):
    sum : float = 0.;
    for i in np.arange(0, nSamples):
        u : float = _rng.uniform()
        x_i = 10. * u + sys.float_info.epsilon * (u == 0)
        sum += integrandFunction(x_i)
    return domainSize * sum / nSamples
\end{minted}
